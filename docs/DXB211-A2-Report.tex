\documentclass[12pt,a4paper]{article}
\usepackage[margin=1in]{geometry}
\usepackage[utf8]{inputenc}
\usepackage{graphicx}
\usepackage{caption}
\usepackage[none]{hyphenat}
\usepackage{hyperref}
\usepackage[nameinlink]{cleveref}

\setlength{\fboxrule}{2pt}

\begin{document}
	\begin{titlepage}
		
		\begin{center}
			\includegraphics[width=0.5\textwidth]{QUT.jpg}\\
			[0.03\textheight]  
			\Large\textbf{Bachelor of IT (Computer Science)}\\
			\Large\textbf{Assignment 2 - Creative Coding Project}\\
			\large\textbf{DXB211 - Creative Coding}\\
			[0.02\textheight]
			\large\textsl{Dane Madsen}\\
			\large\textsl{n10983864@qut.edu.au}
		\end{center}
		
	\end{titlepage}
	\tableofcontents
	\newpage

	\section{Introduction}
		In WWII Germany, Enigma was an instrumental tool in the German war effort. 
		Enigma was a machine used to encrypt and decrypt intelligence communications 
		between German forces. The sketch created for this assignment aims to simulate 
		the Enigma machines encryption and decryption process.\\
		\\
		To run this sketch you will need to run \texttt{python -m http.server} in the 
		\texttt{src} folder of the project. Then navigate to \texttt{localhost:8000} 
		in your web browser and open the \texttt{entry.html} file.\\
		\\
		To use the sketch, set the three rotors to the desired positions, then simply 
		type and plain text will be displayed next to the \texttt{Plain Text} heading along 
		with the encrypted / decrypted text next to the \texttt{Cipher Text} heading.\\

		\begin{center}
			\includegraphics[width=0.8\textwidth]{figures/figure1.jpg}\\
			\vspace{0.5cm}
			\includegraphics[width=0.8\textwidth]{figures/figure2.jpg}\\
		\end{center}
	
	\newpage

	\section{Design and Aesthetic}
		The sketch has been designed to roughly resemble the style of the 
		\href{https://www.asd.gov.au/}{Australian Signals Directorate (ASD)} website. 
		The ASD is the intelligence agency of Australia responsible for conducting 
		signals intelligence on behalf of the Australian Government. As such, 
		Cryptography is highly relevant to the ASD's work. Another source of styling was 
		\href{https://www.qut.edu.au/study/information-technology?undergraduate}{QUT's own website} 
		as QUT also uses alot of dark blue colours like the ASD.\\
		
		\begin{center}
			\fbox{\includegraphics[width=0.8\textwidth]{figures/figure3.jpg}}\\
			\vspace{0.5cm}
			\fbox{\includegraphics[width=0.8\textwidth]{figures/figure4.jpg}}\\
		\end{center}

		I chose to design a P5JS Enigma machine because I have always been interested 
		in Cryptography and the cabinets in the brief reminded me of the Enigma machine. 
		As such, creating a P5JS Enigma machine presented me with an opportunity to 
		both learn more about cryptography whilst also creating an interesting and appealing 
		sketch. I chose rounded boxes for the UI because i thought it made the sketch look 
		more modern. \\

	\newpage

	\section{Design Process}
		In the initial design of the sketch, I planned to have the rotors displayed stacked 
		vertically in the center of the page. In this version of the sketch the actual values 
		of the rotors would be displayed accross the page as strings of text.\\
		
		\begin{center}
			\fbox{\includegraphics[width=0.8\textwidth]{figures/figure5.jpg}}\par
		\end{center}

		I thought this looked interesting but ultimately it served no purpose so I decided to 
		remove it in favor of a more minimalistic design with a simple number to represent 
		each rotor position. At this point I also took the liberty to chnage the arrangement of 
		the rotor position numbers to be side by side at the top of the page.\\

		\begin{center}
			\fbox{\includegraphics[width=0.8\textwidth]{figures/figure6.jpg}}\par
		\end{center}
	
	\newpage

	\section{Creative Influences}

\end{document}